\section{Introduction}

The area of modern neuroscience deals with the understanding of the brain. From special interest is the connection of specific parts of the brain to neurological disorders or cognitive stimuli. 
Based on this, patterns of specific neurological diseases can be detected and a personalized treatment for patients can be developed.

This report summarizes the work by Li et al. \cite{LI2021102233} about "BrainGNN: In\-ter\-pre\-ta\-ble Brain Graph Neural Network for fMRI Analysis".
The work by Li et al. is located in the area of modern neuroscience. 
Through fMRI (functional magnetic resonance imaging) the brain is observed. Based on these fMRI recordings, a graph of the brain is built. The nodes of the graph correspond to different brain regions and the edges describe interactions between different brain regions. This work then proposes a Brain Graph Neural Network (BrainGNN) for graph classification (e.g. is a person healthy or has a specific neurological disease?).
The main novelties proposed of this work are an improved graph convolutional layer and the inbuilt model interpretability.